\documentclass[10pt,a4paper,twocolumn]{article}
\usepackage{times}
\usepackage{amsmath}
\usepackage{amssymb}
\usepackage{siunitx}
\usepackage{hyperref}
\usepackage{xparse}
\usepackage{relsize}
\usepackage{array}
\newcolumntype{C}[1]{>{\centering\let\newline\\\arraybackslash\hspace{0pt}}m{#1}}
\usepackage{tikz,pgfplots}
\usepgfplotslibrary{fillbetween}
\usetikzlibrary{patterns}
\pgfplotsset{compat=1.18}
\usepackage[margin=2cm]{geometry}
\usepackage[most]{tcolorbox}

\definecolor{britishracinggreen}{rgb}{0.0, 0.26, 0.15}

\tcbset{
    frame code={}
    colback=gray!70,
    colframe=white,
    boxsep=0pt,
    enlarge left by=0mm,
}
\newcommand{\capequation}[1]{\begin{tcolorbox} #1 \end{tcolorbox}\vspace{-0.5cm}}

\title{\textbf{Introducción a los Dispositivos Electrónicos [TB064]}}

\begin{document}
\date{} 
\maketitle

\section{Diodos}

\capequation{Caída de tensión en el diodo}
\begin{equation}
    V_D = V_A - V_K
\end{equation}

\capequation{Corriente en el diodo de juntura}
\begin{equation}
   I_D = I_S \cdot \left[\exp \left(\frac{V_D}{nV_{th}} \right)  - 1 \right]
\end{equation}

\capequation{Tensión térmica}
\begin{equation}
   V_{th} = \frac{kT}{q}
\end{equation}

\capequation{Regla de los $60mV$}
\begin{equation}
   I_D \times 10 = V_D + 60mV
\end{equation}

\subsection{Modelo de pequeña señal diodo}
    
\capequation{Caída de tensión en el diodo}
\begin{equation}
    v_D(t) = V_D + v_d(t)
\end{equation}

\capequation{Validez del MPS}
\begin{equation*}
    v_D < 10mV
\end{equation*}

\capequation{Corriente en el diodo}
\begin{equation}
    i_D(t)  = I_D + i_d(t)= I_D + \frac{I_D+I_S}{nVth} \cdot v_d(t)
\end{equation}

\capequation{Conductancia}
\begin{equation}
    g_D = \frac{\partial i_D}{\partial v_D} \Bigg|_Q = \frac{I_D + I_S}{nV_{th}} = \frac{1}{r_D}
\end{equation}

\capequation{Capacidad total del diodo}
\begin{equation}
    C_D = C_{d} + C_{j}
\end{equation}

\capequation{Capacidad de difusión}
\begin{equation}
    C_{d} = \frac{\tau_T \cdot I_D}{nV_{th}}
\end{equation}

\capequation{Capacidad de juntura}
\begin{equation}
    C_{j} = \frac{C_{j0}}{\sqrt{1 - \frac{V_D}{\phi_B}}}
\end{equation}

\section{Transistor Bipolar de Juntura TBJ}
\capequation{Punto de polarización}
\begin{equation}
    Q = (I_C;\; V_{CE})
\end{equation}

\capequation{Corriente de base NPN (con efecto early)}
\begin{equation}
    I_B = \frac{I_S}{\beta} \cdot \exp{\left(\frac{V_{BE}}{nV_{th}}\right)}\cdot \left( 1 + \frac{V_{CE}}{V_A} \right)
\end{equation}

\capequation{Corriente de colector NPN (con efecto early)}
\begin{equation}
    I_C = \beta I_B \cdot \left( 1 + \frac{V_{CE}}{V_A} \right)
\end{equation}

%\textbf{*}\textit{En un TBJ PNP el termino de early esta restando}

\capequation{Condiciones de polarización TBJ NPN}
\begin{equation}
MAD: \quad \begin{cases}
V_{BE} = 0,7 \, V \\
V_{CE} > \, V_{CE(\text{SAT})} = 0,2 V \\
V_{BE} - V_{CE} - V_{BC} = 0
\end{cases}
\end{equation}

\capequation{Condiciones de polarización TBJ PNP}
\begin{equation}
MAD: \quad \begin{cases}
V_{BE} = -0,7 \, V \\
V_{CE} < \, V_{CE(\text{SAT})} = -0,2 V \\
V_{BE} + V_{CE} + V_{BC} = 0
\end{cases}
\end{equation}

\subsection{Modelo de pequeña señal TBJ}

\capequation{Validez del MPS}
\begin{equation}
    v_{be} \leq 10mV
\end{equation}

\capequation{Transconductancia de salida}
\begin{equation}
    g_m = \frac{\partial i_C}{\partial v_{BE}} \Bigg|_Q = \frac{I_{C_Q}}{V_{th}}
\end{equation}

\capequation{Conductancia de salida}
\begin{equation}
    g_o = \frac{\partial i_C}{\partial v_{CE}} \Bigg|_Q = \frac{I_{C_Q}}{V_A}
\end{equation}

\capequation{Conductancia de entrada}
\begin{equation}
    g_{\pi} = \frac{\partial i_B}{\partial v_{BE}} \Bigg|_Q = \frac{g_m}{\beta}
\end{equation}

\capequation{Conductancia de retroalimentación}
\begin{equation}
    g_{\mu} = \frac{\partial i_B}{\partial v_{CE}} \Bigg|_Q \simeq 0
\end{equation}




















\section{MOSFET}
\capequation{Relación entre tensiones}
\begin{equation}
    V_{DS\;(SAT)} = V_{GS} - V_T 
\end{equation}

\capequation{Condiciones de polarización NMOS}

\begin{equation*}
Corte: \quad \begin{cases}
    V_{GS} < V_T \\
    I_D = 0
\end{cases} \\
\end{equation*}
\begin{equation*}
  Triodo:\begin{cases}
    V_{GS} > V_T \\
    V_{DS} < V_{DS\;(SAT)} \\
    I_D = K'_n\frac{W}{L}\left[V_{DS\;(SAT)} - \frac{V_{DS}}{2}\right]V_{DS}
\end{cases} \\  
\end{equation*}

\begin{equation*}
    Saturacion\begin{cases}
    V_{GS} > V_T \\
    V_{DS} > V_{DS\;(SAT)} \\
    I_D = \frac{K'_n}{2}\frac{W}{L}\left[V_{DS\;(SAT)}\right]^2
\end{cases}
\end{equation*}

%%%%%%%%%%%%%%%%%%%%%%%%
\capequation{Condiciones de polarización PMOS}

\begin{equation*}
    Corte\begin{cases}
    V_{GS} > V_T \\
    I_D = 0
\end{cases} \\
\end{equation*}
\begin{equation*}
    Triodo\begin{cases}
    V_{GS} < V_T \\
    V_{DS} > V_{DS\;(SAT)} \\
    I_D = -K'_p\frac{W}{L}\left[V_{DS\;(SAT)} - \frac{V_{DS}}{2}\right]V_{DS}
\end{cases} \\
\end{equation*}


\begin{equation*}
    Saturacion\begin{cases}
    V_{GS} < V_T \\
    V_{DS} < V_{DS\;(SAT)} \\
    I_D = -\frac{K'_p}{2}\frac{W}{L}\left[V_{DS\;(SAT)}\right]^2
\end{cases}
\end{equation*}

\capequation{Efecto de modulación del largo del canal}
\begin{equation*}
    I_D \cdot (1+\lambda V_{DS})
\end{equation*}

\capequation{Back-Gate NMOS}
\begin{equation}
    V_{T} = V_{T0} + \gamma{\left[\sqrt{\left(-2\phi_p - V_{BS}\right)} - \sqrt{\left( -2\phi_p\right)}  \right]} 
\end{equation}

\capequation{Back-Gate PMOS}
\begin{equation}
    V_{T} = V_{T0} - \gamma{\left[\sqrt{2\phi_n - V_{BS}} - \sqrt{2\phi_n}\right]} 
\end{equation}

\subsection{Modelo de pequeña señal MOSFET}

\capequation{Validez del MPS}
\begin{equation*}
    v_{gs} < \frac{V_{GS} - V_T}{5}
\end{equation*}

\capequation{Transconductancia de salida}
\begin{equation}
    g_m = \frac{\partial i_D}{\partial v_{GS}} \Bigg|_Q = K'_n\frac{W}{L}\left(V_{DS(SAT)}\right) \cdot (1+\lambda V_{DS})
\end{equation}

\capequation{Conductancia de salida}
\begin{equation}
    g_o = \frac{\partial i_D}{\partial v_{DS}} \Bigg|_Q = \lambda \cdot I_{D(SAT)}
\end{equation}

\section{Amplificadores}

\subsection{Parámetros característicos}
Parámetros del amplificador sin fuente de señal ni carga. 

\capequation{Ganancia de tensión sin carga}

\begin{itemize}
    \item Se aplica fuente de prueba a la entrada y se saca la  carga.
\end{itemize}
\begin{equation}
    A_{vo} = \frac{v_{out}}{v_{in}}\bigg |_{i_{out} = 0} < 0
\end{equation}

\capequation{Resistencia de entrada}
\begin{itemize}
    \item Se aplica fuente de prueba a la entrada y se saca la  carga.
\end{itemize}
\begin{equation}
    R_{in} = \frac{v_{in}}{i_{in}} \bigg |_{i_{out} = 0}
\end{equation}

\capequation{Resistencia de entrada}
\begin{itemize}
    \item Se reemplaza la carga por una fuente de prueba a la salida y el generador controlado no se enciende.
\end{itemize}
\begin{equation}
    R_{out} = \frac{v_{out}}{i_{out}} \bigg |_{v_{in} = 0}
\end{equation}


\subsection{Parámetros en funcionamiento}
Parámetros del amplificador con carga y fuente de señal. 

\capequation{Ganancia de tensión en funcionamiento}
\begin{itemize}
    \item Se tiene en cuenta la fuente de señal con su resistencia interna y la carga.
\end{itemize}
\begin{equation}
    A_{vs} = \frac{v_{out}}{v_{sig}}\Bigg|_{i_{out} \neq 0 }
\end{equation}

\capequation{Ganancia en tensión con carga}
\begin{itemize}
    \item Es la ganancia de tensión con carga a la salida.
\end{itemize}
\begin{equation}
    |A_{v}| = \bigg|\frac{v_{out}}{v_{in}}\bigg| > 1 
\end{equation}





\subsection{Distorsiones para transistor TBJ}

Si se cumplen las siguientes condiciones el amplificador no distorsiona:

\capequation{Distorsión por alinealidad}
\begin{equation}
    v_{be} \leq 10mV
\end{equation}

\capequation{Distorsión por saturación}
\begin{equation}
    v_{ce} < (V_{CE_Q} - V_{CE_{(SAT)}})
\end{equation}

\capequation{Distorsión por corte}
\begin{equation}
    v_{ce} < (V_{CC} - V_{CE_Q})
\end{equation}

\subsection{Distorsiones para transistor MOSFET}

Si se cumplen las siguientes condiciones el amplificador no distorsiona:

\capequation{Distorsión por alinealidad}
\begin{equation}
   v_{gs} \leq 0,2\cdot(V_{GS} - V_T)
\end{equation}

\capequation{Distorsión por corte}
\begin{equation}
    v_{out} < I_{D_Q}\cdot R_D
\end{equation}

\capequation{Distorsión por triodo}
\begin{equation}
    v_{out} < V_{DS_{(SAT)}} = V_{GS} - V_T
\end{equation}








\end{document}