\section{Transistor Bipolar de Juntura TBJ}
\capequation{Punto de polarización}
\begin{equation}
    Q = (I_C;\; V_{CE})
\end{equation}

\capequation{Corriente de base NPN (con efecto early)}
\begin{equation}
    I_B = \frac{I_S}{\beta} \cdot \exp{\left(\frac{V_{BE}}{nV_{th}}\right)}\cdot \left( 1 + \frac{V_{CE}}{V_A} \right)
\end{equation}

\capequation{Corriente de colector NPN (con efecto early)}
\begin{equation}
    I_C = \beta I_B \cdot \left( 1 + \frac{V_{CE}}{V_A} \right)
\end{equation}

%\textbf{*}\textit{En un TBJ PNP el termino de early esta restando}

\capequation{Condiciones de polarización TBJ NPN}
\begin{equation}
MAD: \quad \begin{cases}
V_{BE} = 0,7 \, V \\
V_{CE} > \, V_{CE(\text{SAT})} = 0,2 V \\
V_{BE} - V_{CE} - V_{BC} = 0
\end{cases}
\end{equation}

\capequation{Condiciones de polarización TBJ PNP}
\begin{equation}
MAD: \quad \begin{cases}
V_{BE} = -0,7 \, V \\
V_{CE} < \, V_{CE(\text{SAT})} = -0,2 V \\
V_{BE} + V_{CE} + V_{BC} = 0
\end{cases}
\end{equation}

\subsection{Modelo de pequeña señal TBJ}

\capequation{Validez del MPS}
\begin{equation*}
    v_{be} < 10mV
\end{equation*}

\capequation{Transconductancia de salida}
\begin{equation}
    g_m = \frac{\partial i_C}{\partial v_{BE}} \Bigg|_Q = \frac{I_{C_Q}}{V_{th}}
\end{equation}

\capequation{Conductancia de salida}
\begin{equation}
    g_o = \frac{\partial i_C}{\partial v_{CE}} \Bigg|_Q = \frac{I_{C_Q}}{V_A}
\end{equation}

\capequation{Conductancia de entrada}
\begin{equation}
    g_{\pi} = \frac{\partial i_B}{\partial v_{BE}} \Bigg|_Q = \frac{g_m}{\beta}
\end{equation}

\capequation{Conductancia de retroalimentación}
\begin{equation}
    g_{\mu} = \frac{\partial i_B}{\partial v_{CE}} \Bigg|_Q \simeq 0
\end{equation}



















