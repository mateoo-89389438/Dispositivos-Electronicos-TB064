\section{Amplificadores}

\subsection{Parámetros característicos}
Parámetros del amplificador sin fuente de señal ni carga. 

\capequation{Ganancia de tensión sin carga}

\begin{itemize}
    \item Se aplica fuente de prueba a la entrada y se saca la  carga.
\end{itemize}
\begin{equation}
    A_{vo} = \frac{v_{out}}{v_{in}}\bigg |_{i_{out} = 0} < 0
\end{equation}

\capequation{Resistencia de entrada}
\begin{itemize}
    \item Se aplica fuente de prueba a la entrada y se saca la  carga.
\end{itemize}
\begin{equation}
    R_{in} = \frac{v_{in}}{i_{in}} \bigg |_{i_{out} = 0}
\end{equation}

\capequation{Resistencia de entrada}
\begin{itemize}
    \item Se reemplaza la carga por una fuente de prueba a la salida y el generador controlado no se enciende.
\end{itemize}
\begin{equation}
    R_{out} = \frac{v_{out}}{i_{out}} \bigg |_{v_{in} = 0}
\end{equation}


\subsection{Parámetros en funcionamiento}
Parámetros del amplificador con carga y fuente de señal. 

\capequation{Ganancia de tensión en funcionamiento}
\begin{itemize}
    \item Se tiene en cuenta la fuente de señal con su resistencia interna y la carga.
\end{itemize}
\begin{equation}
    A_{vs} = \frac{v_{out}}{v_{sig}}\Bigg|_{i_{out} \neq 0 }
\end{equation}

\capequation{Ganancia en tensión con carga}
\begin{itemize}
    \item Es la ganancia de tensión con carga a la salida.
\end{itemize}
\begin{equation}
    |A_{v}| = \bigg|\frac{v_{out}}{v_{in}}\bigg| > 1 
\end{equation}





\subsection{Distorsiones para transistor TBJ}

Si se cumplen las siguientes condiciones el amplificador no distorsiona:

\capequation{Distorsión por alinealidad}
\begin{equation}
    v_{be} \leq 10mV
\end{equation}

\capequation{Distorsión por saturación}
\begin{equation}
    v_{ce} < (V_{CE_Q} - V_{CE_{(SAT)}})
\end{equation}

\capequation{Distorsión por corte}
\begin{equation}
    v_{ce} < (V_{CC} - V_{CE_Q})
\end{equation}

\subsection{Distorsiones para transistor MOSFET}

Si se cumplen las siguientes condiciones el amplificador no distorsiona:

\capequation{Distorsión por alinealidad}
\begin{equation}
   v_{gs} \leq 0,2\cdot(V_{GS} - V_T)
\end{equation}

\capequation{Distorsión por corte}
\begin{equation}
    v_{out} < I_{D_Q}\cdot R_D
\end{equation}

\capequation{Distorsión por triodo}
\begin{equation}
    v_{out} < V_{DS_{(SAT)}} = V_{GS} - V_T
\end{equation}





